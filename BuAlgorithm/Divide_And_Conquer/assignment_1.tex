%%%%%%%%%%%%%%%%%%%%%%%%%%%%%%%%%%%%%%%%%
% Programming/Coding Assignment
% LaTeX Template
%
% This template has been downloaded from:
% http://www.latextemplates.com
%
% Original author:
% Ted Pavlic (http://www.tedpavlic.com)
%
% Note:
% The \lipsum[#] commands throughout this template generate dummy text
% to fill the template out. These commands should all be removed when 
% writing assignment content.
%
% This template uses a Python script as an example snippet of code, most other
% languages are also usable. Configure them in the "CODE INCLUSION 
% CONFIGURATION" section.
%
%%%%%%%%%%%%%%%%%%%%%%%%%%%%%%%%%%%%%%%%%

%----------------------------------------------------------------------------------------
%	PACKAGES AND OTHER DOCUMENT CONFIGURATIONS
%----------------------------------------------------------------------------------------

\documentclass{article}

\usepackage{fancyhdr} % Required for custom headers
\usepackage{lastpage} % Required to determine the last page for the footer
\usepackage{extramarks} % Required for headers and footers
\usepackage[usenames,dvipsnames]{color} % Required for custom colors
\usepackage{graphicx} % Required to insert images
\usepackage{listings} % Required for insertion of code
\usepackage{courier} % Required for the courier font
\usepackage{amsmath}
\usepackage{enumerate}
\usepackage{algorithm}
% Margins
\topmargin=-0.45in
\evensidemargin=0in
\oddsidemargin=0in
\textwidth=6.5in
\textheight=9.0in
\headsep=0.25in

\linespread{1.3} % Line spacing

% Set up the header and footer
\pagestyle{fancy}
\lhead{\hmwkAuthorName} % Top left header
\chead{\hmwkClass\ (\hmwkClassInstructor\ \hmwkClassTime): \hmwkTitle} % Top center head
\rhead{\firstxmark} % Top right header
\lfoot{\lastxmark} % Bottom left footer
\cfoot{} % Bottom center footer
\rfoot{Page\ \thepage\ of\ \protect\pageref{LastPage}} % Bottom right footer
\renewcommand\headrulewidth{0.4pt} % Size of the header rule
\renewcommand\footrulewidth{0.4pt} % Size of the footer rule

\setlength\parindent{0pt} % Removes all indentation from paragraphs




%----------------------------------------------------------------------------------------
%   CODE INCLUSION CONFIGURATION
%----------------------------------------------------------------------------------------

\definecolor{MyDarkGreen}{rgb}{0.0,0.4,0.0} % This is the color used for comments
\lstloadlanguages{Python} % Load Perl syntax for listings, for a list of other languages supported see: ftp://ftp.tex.ac.uk/tex-archive/macros/latex/contrib/listings/listings.pdf
\lstset{language=Python, % Use Perl in this example
        %frame=single, % Single frame around code
        basicstyle=\small\ttfamily, % Use small true type font
        keywordstyle=[1]\color{Blue}\bf, % Perl functions bold and blue
        keywordstyle=[2]\color{Purple}, % Perl function arguments purple
        keywordstyle=[3]\color{Blue}\underbar, % Custom functions underlined and blue
        identifierstyle=, % Nothing special about identifiers                                         
        commentstyle=\usefont{T1}{pcr}{m}{sl}\color{MyDarkGreen}\small, % Comments small dark green courier font
        stringstyle=\color{Purple}, % Strings are purple
        showstringspaces=false, % Don't put marks in string spaces
        tabsize=3, % 5 spaces per tab
        %
        % Put standard Perl functions not included in the default language here
        morekeywords={rand},
        %
        % Put Perl function parameters here
        morekeywords=[2]{on, off, interp},
        %
        % Put user defined functions here
        morekeywords=[3]{test},
        %
        morecomment=[l][\color{Blue}]{...}, % Line continuation (...) like blue comment
        numbers=left, % Line numbers on left
        firstnumber=1, % Line numbers start with line 1
        numberstyle=\tiny\color{Blue}, % Line numbers are blue and small
        stepnumber=5 % Line numbers go in steps of 5
}

% Creates a new command to include a python script, the first parameter is the filename of the script (without .py), the second parameter is the caption
\newcommand{\pythonscript}[2]{
\begin{itemize}
\item[]\lstinputlisting[caption=#2,label=#1]{#1.py}
\end{itemize}
}

%----------------------------------------------------------------------------------------
%	NAME AND CLASS SECTION
%----------------------------------------------------------------------------------------

\newcommand{\hmwkTitle}{Assignment\ \#1} % Assignment title
\newcommand{\hmwkDueDate}{Thursday,\ Oct \ 8,\ 2015} % Due date
\newcommand{\hmwkClass}{CS091M4041H\ 101} % Course/class
\newcommand{\hmwkClassTime}{10:30am} % Class/lecture time
\newcommand{\hmwkClassInstructor}{Dongbo Bu} % Teacher/lecturer
\newcommand{\hmwkAuthorName}{cwlseu} % Your name

%----------------------------------------------------------------------------------------
%	TITLE PAGE
%----------------------------------------------------------------------------------------

\title{
\vspace{2in}
\textmd{\textbf{\hmwkClass:\ \hmwkTitle}}\\
\normalsize\vspace{0.1in}\small{Due\ on\ \hmwkDueDate}\\
\vspace{0.1in}\large{\textit{\hmwkClassInstructor\ \hmwkClassTime}}
\vspace{3in}
}

\author{\textbf{\hmwkAuthorName}\\\textbf{\hmwkAuthorStuID}}
\date{} % Insert date here if you want it to appear below your name

%----------------------------------------------------------------------------------------

\begin{document}

\maketitle
\newpage
%----------------------------------------------------------------------------------------
%	TABLE OF CONTENTS
%----------------------------------------------------------------------------------------

%\setcounter{tocdepth}{1} % Uncomment this line if you don't want subsections listed in the ToC

%----------------------------------------------------------------------------------------
%	PROBLEM 1
%----------------------------------------------------------------------------------------

% To have just one problem per page, simply put a \clearpage after each problem
\section{Divide and Conquer-1}
    You are interested in analyzing some hard-to-obtain data from two separate databases. Each
    database contains $n$  numerical values, so there are $2n$  values total and you may assume that no
    two values are the same. You’d like to determine the median of this set of $2n$ values, which we
    will define here to be the $n^{th}$ smallest value.\\
    However, the only way you can access these values is through queries to the databases. In
    a single query, you can specify a value k to one of the two databases, and the chosen database
    will return the $k^{th}$ smallest value that it contains. Since queries are expensive, you would liketo compute the median using as few queries as possible. \\
    Give an algorithm that finds the median value using at most $O(logn)$ queries.
    \subsection{Algorithm}
    Input two arrays: $An$ and $Bn$. (each array contains $n$ different value and is in \emph{ascending} order).The system will return result is the $k^{th}$ smallest value that it contains.
    So we get the median of two arrays: $A[\frac{n}{2}]$ and $B[\frac{n}{2}]$. Before the $A[\frac{n}{2}]$ value, there are only $\frac{n}{2}-1$ values, before the $B[\frac{n}{2}]$ value, there are only $\frac{n}{2}-1$ values. (\emph{Note}: $A[\frac{n}{2}] \neq B[\frac{n}{2}]$  because all $2n$ values are different). \\
    So we only have two conditions:
    \begin{itemize}
    \item if $A[\frac{n}{2}] < B[\frac{n}{2}]$, The two arrays’ median value must be in the interval $A[\frac{n}{2}+1]$ to $A[n]$ and in the interval B[1] to $B[\frac{n}{2}]$. 
    \item if $A[\frac{n}{2}] > B[\frac{n}{2}]$ ,the two arrays’ median value must be in the interval $A[0]$ to $A[\frac{n}{2}]$ and in the interval $B[\frac{n}{2}+1]$ to $B[n]$.
    \end{itemize}
    
    \pythonscript{getmedian}{Get the median value for two ordered array}
    \subsection{Subproblem Graph}
        \begin{center}
            \includegraphics[width=0.75\columnwidth]{median} % Example image
        \end{center}
    \subsection{Correctness}
    The algorithm get the median of two arrays: $A[\frac{n}{2}]$ and $B[\frac{n}{2}]$ which
     is in ascending order. Before the $A[\frac{n}{2}]$ value, there are only  $\frac{n}{2}-1$ values, before the $B[\frac{n}{2}]$ value, there are only $\frac{n}{2}-1$ values. 
     (\emph{Note}: $A[\frac{n}{2}]\neqB[\frac{n}{2}]$  because all $2n$ values are different). if $A[\frac{n}{2}] < B[\frac{n}{2}]$, before the $B[\frac{n}{2}]$ value, there must be at least $\frac{n}{2}-1 + \frac{n}{2}-1+1 = n-1$ values, which is $\frac{n}{2}-1$ before $A[\frac{n}{2}]$ and $\frac{n}{2}-1$ before $B[\frac{n}{2}]$ and  1  stand for $A[\frac{n}{2}]$  itself. While before the $A[\frac{n}{2}]$ value, there must be at most $\frac{n}{2}-1+\frac{n}{2}-1=n-2$. So the two arrays’ median value must be in the interval $A[\frac{n}{2}+1]$ to $A[n]$ and in the interval $B[0]$ to $B[\frac{n}{2}]$. We find the median value from  subarray and we can divide this problem into one small problem.\\At last, Subarray A and B only have one value, we get the small one which is the result. 
    \subsection{Time Complexity}
    Now, We will prove this algorithm is $O(log n)$ time.  
    In algorithm: we can have $$T(n)=T(\frac{n}{2})+C$$ and when $n>1$ $$T(1)=C$$ So we get $$T(n)=clog n$$
    Then this algorithm is $O(log n)$ time.
\section{Divide and Conquer-3}
    Recall the problem of finding the number of inversions. As in the course, we are given a sequence of n numbers $a_1$,..., $a_n$, which we assume are all distinct, and we difine an inversion to be a pair $i < j$ such that $a_i > a_j$.                                                                               \\
    We motivated the problem of counting inversions as a good measure of how different two orderings are. However, one might feel that this measure is too sensitive. Let’s call a pair a significant inversion if $i < j$ and $a_i > 3a_j$.                                                                                \\
    Given an $O(n log n)$ algorithm to count the number of significant inversions between two orderings.  \\
    \subsection{Algorithm}
    \pythonscript{significant_inversion}{Get a number list significant inversion}
    \subsection{Subproblem Graph}
        \begin{center}
            \includegraphics[width=0.75\columnwidth]{inversions} % Example image
        \end{center}
    \subsection{Correctness}
    Given a sequence of n numbers , we use a array $L[1..n]$ which contains this $n$ numbers. 
    We can use Sort\_And\_Count function in the normal InversionCount. Assume we divide $L[1..n]$ into two subarray $A[0..\frac{n}{2}]$ for $L[0..\frac{n}{2}]$  and $B[0..\frac{n}{2}]$ for $A[\frac{n}{2}+1..n]$,and we have already sort the two arrays A and B and computed the inversions’ numbers.
     Now what we only should do is to get the $A[i]$ and $B[j]$ (i and j is $1..\frac{n}{2}$) and look whether this two number is inversed and at the same time we should merge this two sorted array. 
    Now let us describe this problem: we already get two sorted arrays $A[0..\frac{n}{2}]$ and 
    $B[0..\frac{n}{2}]$, we set three indexes i and $j_1$ and $j_2$ (i and $j_2$ is to iterate A and B, $j_1$ is to $B[j_1..j_2 - 1] < A[i]$  
    in other word, it means the current \emph{ith} value is bigger than $j_1$ to $j_2 - 1$’s all values.) 
    When we iterate i’th number in A and j’th number in B, we have two conditions(because all numbers are distinct, so $A[i]\neq B[j_2]$):
    \begin{itemize}
        \item if $A[i] < B[j_2]$. We look whether $A[i]$is treble than $B[j_1]$ , 
        if $A[i] > 3*B[j_1]$ then we add InverseCount( InverseCount stand for the current number of inversions) with the number of A remaining arrays. At the same time, we add $j_1$ to look the next element in B. 
        Else, we insent $A[i]$ value into new sorted array and add i to look the next element in A.   
        \item if $A[i]>B[j_2]$. We insert $B[j_2]$ value into new new sorted array and add $j_2$ to look the next element in B.
    \end{itemize}
    \subsection{Time Complexity}
    Now, We will prove this algorithm is $O(n log n)$ time.  
    In algorithm: we can have $$T(n)=2T(\frac{n}{2})+ O(n)$$ and when $n>1$ $$T(1)=C$$ So we get $$T(n)=cn log n$$ 
    Then this algorithm is $O(n log n)$ time.

\section{Divide and Conquer-6}
Given a convex polygon with n vertices, we can divide it into several separated pieces, such that every piece is a triangle. When n = 4, there are two different ways to divide the polygon; When n = 5, there are five different ways.
Give an algorithm that decides how many ways we can divide a convex polygon with n vertices into triangles.
    \subsection{Algorithm}
        \pythonscript{DivideConvexPolygon}{Divide a convex polygon with n vertices into triangles}
    \subsection{Reduction Graph}
        \begin{center}
            \includegraphics[width=0.75\columnwidth]{reduce2}
        \end{center}
    \subsection{Correctness}
    The convex polygon with n vertices. We give a sequance number $V_1..V_i..V_n$. As we all know, $V_1$, $V_i$, $V_n$ could consititude a triangle $S$. And the convex polygon have been separated to three parts as following figure show: 
    \begin{center}
    \includegraphics[width=0.25\columnwidth]{polygon}
    \end{center}
    We can see that, the number of divide the convex polygon with n vertices into triangles is  $$f_i(n)=f(i)f(n+1-i)$$ We find the convex polygon divide problem could be divided into two small problem. \\
    At last, Subproblem is a triangles, then the backtrack the problem will get the result.Finally, the totel number is the sum of all the i value such as: $$f(n)=\sum_{i=2}^{n-1}f(i)f(n+1-i)$$
    \subsection{Time Complexity}
    From the Reduction graph, We can see following $$T(n) = \sum_{i=2}^{n-1}T(i)T(n+1-i)$$ The time complexity is $O(2^n)$.The algorithm will be useless for real world testing. If we save the interval value of subproblem, we should use n external space and the $$T(n) = \sum_{i=2}^{n-1}T(i)$$ Then this algorithm is $O(n)$ time.

\section{Divide and Conquer-8}
    Implement the algorithm for the closest pair problem in your favourite language.\\
    \emph{INPUT}: Given n points in a plane.\\
    \emph{OUTPUT}: the pair with the least Euclidean distance.
    \subsection{Code}
        \pythonscript{nearestpair}{algorithm for the closest pair problem in your favourite language}
\section{Divide and Conquer-9}
    Implement the Strassen algorithm algorithm for MatrixMultiplication problem in your favourite
    language, and compare the performance with grade-school method.
    \subsection{Code}
        \pythonscript{strassen}{algorithm for the closest pair problem}
        \emph{There are an Input Case input data file name input.txt} \\
        5\\
        1 2 3 4 5 \\
        1 2 3 4 5 \\
        1 2 3 4 5 \\
        1 2 3 4 5 \\
        1 2 3 4 5 \\
        5\\
        1 2 3 4 5 \\
        1 2 3 4 5 \\
        1 2 3 4 5 \\
        1 2 3 4 5 \\
        1 2 3 4 5 
    \subsection{Performance Compare}
    I use test matrix dimension 5, 20, 60, 160, 320. Every method running 10 time and get the average time as $Table 1$ shows:
      \begin{table}[tbp]     
            \centering 
                \begin{tabular}{ccc}  
                \hline
                    Matrix Dimension & Grade-school & Strassen  \\ \hline  % \hline 在此行下面画一横线
                    5          & 0.000045     & 0.000064  \\
                    20         & 0.002432     & 0.001844  \\
                    60         & 0.045801     & 0.044032  \\
                    160        & 0.890432     & 0.740964  \\
                    320        & 7.210309     & 5.449172  \\ \hline
                \end{tabular}
            \caption{Average running time with different method.}
        \end{table}
    From test result, I find that if the matrix is smaller, the grade-school method is better than Strassen. So, When I implement the Strassen, I use a commend to decide use grade-school method or divide the matrix to small matrix. 
%----------------------------------------------------------------------------------------
\end{document}