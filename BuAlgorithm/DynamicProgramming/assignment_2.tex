% !Mode:: "TeX:UTF-8"
\documentclass{article}
\usepackage{ctex}
\usepackage{lastpage} % Required to determine the last page for the footer
\usepackage{extramarks} % Required for headers and footers
\usepackage{graphicx} % Required to insert images
\usepackage{listings} % Required for insertion of code
\usepackage[usenames,dvipsnames]{color} % Required for custom colors
\usepackage{amsmath,amsthm}
\usepackage{courier} % Required for the courier font
\usepackage{enumerate}
\usepackage{algorithm}
\usepackage{algorithmic}
\usepackage{algpseudocode}
% Margins
\linespread{1.3} % Line spacing
% Set up the header and footer

%----------------------------------------------------------------------------------
%   CODE INCLUSION CONFIGURATION
%----------------------------------------------------------------------------------------
\definecolor{MyDarkGreen}{rgb}{0.0,0.4,0.0} % This is the color used for comments
\lstloadlanguages{Python} % Load Perl syntax for listings, for a list of other languages supported see: ftp://ftp.tex.ac.uk/tex-archive/macros/latex/contrib/listings/listings.pdf
\lstset{language=Python, % Use Perl in this example
        %frame=single, % Single frame around code
        basicstyle=\small\ttfamily, % Use small true type font
        keywordstyle=[1]\color{Blue}\bf, % Perl functions bold and blue
        keywordstyle=[2]\color{Purple}, % Perl function arguments purple
        keywordstyle=[3]\color{Blue}\underbar, % Custom functions underlined and blue
        identifierstyle=, % Nothing special about identifiers
        commentstyle=\usefont{T1}{pcr}{m}{sl}\color{MyDarkGreen}\small, % Comments small dark green courier font
        stringstyle=\color{Purple}, % Strings are purple
        showstringspaces=false, % Don't put marks in string spaces
        tabsize=3, % 5 spaces per tab
        %
        % Put standard Perl functions not included in the default language here
        morekeywords={rand},
        %
        % Put Perl function parameters here
        morekeywords=[2]{on, off, interp},
        %
        % Put user defined functions here
        morekeywords=[3]{test},
        %
        morecomment=[l][\color{Blue}]{...}, % Line continuation (...) like blue comment
        numbers=left, % Line numbers on left
        firstnumber=1, % Line numbers start with line 1
        numberstyle=\tiny\color{Blue}, % Line numbers are blue and small
        stepnumber=5 % Line numbers go in steps of 5
}

% Creates a new command to include a python script, the first parameter is the filename of the script (without .py), the second parameter is the caption
\newcommand{\pythonscript}[2]{
\begin{itemize}
\item[]\lstinputlisting[caption=#2,label=#1]{#1.py}
\end{itemize}
}

%----------------------------------------------------------------------------------------
%   NAME AND CLASS SECTION
%----------------------------------------------------------------------------------------

\newcommand{\hmwkTitle}{Assignment\ \#2} % Assignment title
\newcommand{\hmwkDueDate}{Thursday,\ Oct \ 8,\ 2015} % Due date
\newcommand{\hmwkClass}{CS091M4041H\ 101} % Course/class
\newcommand{\hmwkClassTime}{9:20am} % Class/lecture time
\newcommand{\hmwkClassInstructor}{Dongbo Bu} % Teacher/lecturer
\newcommand{\hmwkAuthorName}{cwlseu} % Your name

\numberwithin{equation}{section}
%----------------------------------------------------------------------------------------
%   TITLE PAGE
%----------------------------------------------------------------------------------------

\title{
\vspace{2in}
\textmd{\textbf{\hmwkClass:\ \hmwkTitle}}\\
\normalsize\vspace{0.1in}\small{Due\ on\ \hmwkDueDate}\\
\vspace{0.1in}\large{\textit{\hmwkClassInstructor\ \hmwkClassTime}}
\vspace{3in}
}

\author{\textbf{\hmwkAuthorName}\\\textbf{\hmwkAuthorStuID}}
\date{} % Insert date here if you want it to appear below your name

\begin{document}

\maketitle
\newpage


%
% Problem 1
%
\section{Minimum Path Sum}
Find the mininum path sum of a triangle from top to bottom. Each step you mey move to adjacent numbers on the row below.
	\subsection{Algorithm}
	The input triangle could be seen as an tree structure. Suppose $x$ and $y$ are children of $k$. Once minimum paths from $x$ and $y$ to the bottom are known, the minimum path starting from $k$ can be decided in $O(1)$, that is optimal substructure. Therefore, dynamic programming would be the best solution to this problem.\\
	\emph{Assumption:} The triangle save in an array $M[n][n]$ and $i$ as interator for lines and $j$ as colum interator.So,in the array all the data save in the left bottom side of the matrix(2-dim-array).\
	In this case, if we are standing in a specific point and since we can only go down and right we can disregard all the cells above and to the left when looking for the minimal solution.
	It's obvious that $$f(i,j) = min\{f(i+1,j),f(i+1,j+1)\} + M[i][j]$$. However, if we look closely, we would notice that the adjacent nodes always share a branch. In other word, there are overlapping subproblems. So, we can use an extend space $OPT$ to save interval calculate values.
		\alglanguage{pseudocode}
		\begin{algorithm}
			\caption{MiniSumPath}
			\label{alg1}
			\begin{algorithmic}[1]
			\REQUIRE $M$ save the triangle
			\REQUIRE $OPT$ save the DP internal value
			\Function{MiniSumPath}{M}
			\STATE $high \gets M.size$
			\STATE $OPT \gets 0x7FFFFFFF$
			\FOR{$i \gets high-1, i \geq 1$}
				\FOR{$j \gets 1, j \leq i$}
					\STATE $OPT[j] = min\{OPT[j],OPT[j+1]\} + M[i][j]$ 
				\ENDFOR
			\ENDFOR
			%\STATE $OPT[i][j] \gets min\{MiniSumPath(i-1,j),MiniSumPath(i-1,j-1)\} + M[i][j]$
			\RETURN $OPT[0]$
			\EndFunction			
			\end{algorithmic}
		\end{algorithm}
	\subsection{Correctness}
	In this algorithm, I use a extra space $OPT$ to save line $i$ all element get down to bottom minimal sum at least.
	I calculate from bottom of triangle, in other word, line $\{hight\ of \ triangle\} - 1$. Step by step, we could get the minimal sum path from top of triangle to bottom.
	\subsection{Complexity}
	The algorithm use array $OPT$ to save interval value, from bottom to up to get minimal sum path to bottom iteratal. Then the time complexity is $O(n^2)$


%
% Problem 2
%
\section{Subsequnce Counting}
Given a string S and a string T, counting the number of distinct subsequences of T in S.
	\subsection{Algorithm}
	In this problem, we can think a scenario that, the $S[S.length] \eq T[T.length]$, 
	then we could delete the last charater from T or keep it. The totle number of distinct subsequnces of $T$ in $S$ is the sum of the number of distinct subsequnces of $T[1..T.length-1]$ in $S[1..S.length-1]$ and the number of distinct subsequnces of $T[1..T.length]$ in $S[1..S.length-1]$.
	If the $S[S.length] \beq T[T.length]$, then the T should be find in subsequnece string of $S$,$S[1..S.length-1]$.
	\emph{Assumption:} $d[i][j]$ is the number of distinct subsequences of string $T[1..i]$ and $S[1..j]$. Above that, we can get optimal substructure. Therefore, DP could sovle this problem. We use $d[i][j]$ save the number f distinct subsequeces of $T[1..i]$ and $S[1..j]$. 
	 So the DP equation as following:
	$$d[i][j] = \begin{cases}
		d[i-1][j-1] + d[i][j-1] & \text{if $T[i] \eq S[j]$},\\
		d[i][j-1]               & \text{if $T[i] \neq S[j]$}.
				\end{cases}
	$$
	Finally the program returns $d[T.length][S.length]$. \\
		\begin{algorithm}
		\caption{Subsequnce Counting}
		\label{alg2}
		\begin{algorithmic}[1]
		\Function{SubseqCount}{T, S}
			\FOR{$j \gets 1, j \leq S.length$}
				\STATE $d[0][j] = 1$
			\ENDFOR
			\FOR{$i \gets 1, i \leq T.length$}
				\FOR{$j \gets 1, j \leq S.length$}
					\IF{$T[i] \eq S[j]$}
						\STATE $d[i][j] = d[i-1][j-1] + d[i][j-1]$ \COMMENT{Keep the S[j] or not}
					\ELSE
						\STATE $d[i][j] = d[i][j-1]$
					\ENDIF
				\ENDFOR
			\ENDFOR
			\RETURN $d[T.length][S.length]$
		\EndFunction
		  \end{algorithmic}
		\end{algorithm}
	\subsection{Correctness}
	The algorithm calculate from empty string T, adding character one by one, to get the number of distinct of 
	subsequnce string $T[1..i]$ in $S[1..j]$. Save the interval result to $d[i][j]$. We can see that, the algorithm
	get the result from up to bottom to get distinct number of all the possible combination of $T$ and $S$. So, the finally result is $d[T.length][S.length]$.
	\subsection{Complexity}	
	From the algorithm, we should calculte $n^2$ values of $d$, every value use constant time. So, the time complexity of the alogrithm should be $O(n^2)$



%
% Problem 3
%
\section{Decoding}
	A message containing letters from $A-Z$ is being encoded to numbers using the following mapping:\\
	$$A:1$$ $$B:2$$ $$C:3$$ $$...$$ $$Z:26$$ 
	Given an encoded message containing digits, determine the totle number of ways to encode it.
	\subsection{Algorithm}
	We can assumpt the message is a string $a_1a_2a_3..a_n$, the totle number of ways to encode is $f(n)$. Then, we can convert the problem to the last digit $a_{n}$. It's encoded alone or encoded with $a_{n-1}$. So, the DP state translation function as following:
		$$f(n) = f(n-1) + f(n-2) $$
	Note: If the $a_{n-1}a_n$ is encoded together, the number $a_{n-1}a_n$ must be in $[1..26]$.
	\begin{algorithm}
		\caption{Calculate the totle number of encode a digits string}
		\label{alg3}
		\begin{algorithmic}[1]
		\REQUIRE $f \gets {0}$
		\REQUIRE $f[1] \gets 1$
		\Function{DecodingStr}{Str, l} \Comment{String and String length l} 
			\IF{$f[l] \neq 0$}
				\RETURN $f[l]$
			\ENDIF
			\IF{$Str[n-1..n].Convert2Digit in [1..26]$}
				\RETURN $DecodingStr(Str,l-1) + DecodingStr(l-2)$
			\ELSE
				\RETURN $DecodingStr(Str,l-1)$
			\ENDIF
		\EndFunction
		 \end{algorithmic}
	\end{algorithm}
	\subsection{Correctness}
	The original problem can be solved by calling $DecodingStr(Str, len(Str))$. When we get a message string $a_1a_2a_3..a_n$, we can split the string to a substring $a_1a_2a_3..a_{n-1}$. If
	we can calculte the totle number of encoding of the substring, then we take the last digit into account. The last digit is encoded alone or encoded with the last digit of substring.
	\subsection{Complexity}
	The algorithm time complexity is $O(n)$. Also we can use $T(n) = T(n-1) + T(n-2)$. The number of subproblem is exponential. So we use a extention array to save interval result.because of  a lots of recompution. After that, every i in [1..n] $f(i)$ only calculate one time. So the time complexity is  reduced to $O(n)$

%
% Proble 6
%
\section{Maximum Profit of Transactions}
Say you have an array for which the $i$-th element is the price of a given stock on day $i$. Design an algorithm and implement it to find the maximum profit. You may complete at most two transactions.
\subsection{Source Code}
	\emph{INPUT} A file inculding a sequence of transaction, every stocks value is integer.\\
	 translation.txt  $\gets$ 1 3 4 4 5 6 7 8 9 7 6 4 3 2 1 6 8 \\
	\emph{OUTPUT} The maximum profit of transactions
	\pythonscript{maxProfit}{Algorithm for maximum profit of transactions}
	The following picture is a result screensplot.
		\begin{center}
            \includegraphics[width=0.70\columnwidth]{result.jpg} % Example image
        \end{center}
\end{document}
